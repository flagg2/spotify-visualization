\documentclass[a4paper,11pt]{article}
\usepackage[margin=2.5cm]{geometry}
\usepackage{graphicx}
\usepackage{float}
\usepackage{hyperref}
\usepackage{setspace}
\usepackage{caption}
\usepackage{booktabs}

\setstretch{1.15}

\title{Spotify Artist Popularity and Collaboration Visualization}
\author{\textbf{Team:} Samuel Wittlinger, Matej Kaňuch}
\date{Autumn 2025}

\begin{document}
\maketitle

\section{Motivation}
We chose this topic because music is an integral part of everyday life. Music is also strongly connected to artists whose popularity is frequently discussed and compared.

An interesting aspect of song popularity is its temporal behavior. Some artists are \emph{seasonal}, reaching peak popularity only during certain periods. Another important factor is collaboration: Is it a guarantee of success?

We decided to analyze these relationships and trends and translate them into an interactive visualization. In addition to popularity metrics, Spotify provides rich audio features for each track, which allowed us to explore also music sounds of artists.

\section{Data Sources and Preprocessing}
The data used in this project comes from the Spotify platform. Historic data comes from a kaggle dataset, from a user that has collected it for the past two years. During preprocessing, individual tracks were grouped by artist. To keep the visualization readable, the data was filtered by country. We show the data for Slovakia and the Czech Republic.

To visualize collaborations, links were created between artists based on shared tracks. The strength of the link is determined by how often the artists collaborate. This creates an ellaborate net which groups artists by a mix of genre, label and similarity. Additionally, we computed charts over time for the number of songs in the Top 50 charts and and their best ranking one.

Finally, audio features of individual tracks were averaged per artist, creating a compact audio fingerprint that characterizes each artist's overall musical style.

\section{Design Choices}
We opted for a more complex visualization format: a dynamic network graph. In this graph, nodes represent artists, while edges between them represent collaborations. 

The right side of the interface contains an information panel that updates based on user interaction. Either by clicking on an artist in the graph or by searching for them by name. This panel displays detailed information about the selected artist, including: artist photo and ranking within SK or CZ, a direct link to the artist's Spotify profile, an audio fingerprint visualization, a line chart showing the artist's best monthly Top 50 rank, a bar chart showing the number of tracks in the Top 50 per month. 

Users can switch between Slovak and Czech statistics, allowing easy comparison and filtering.

\section{Interesting Observations}
The visualization shows several interesting patterns. Highly connected artists in the collaboration graph often achieve higher and more stable chart rankings. Seasonal artists were clearly visible through peaks and drops in the monthly ranking charts, while consistent artists showed smoother trends.

We also observed that collaborations with already well-established artists often led to temporary popularity boosts, but there are also artists from music styles where collaborations are not very common and they are successful, too.

Moreover, we noticed that even though the popularity of artists between SK and CZ is somewhat correlated, there are some artists which are very popular in one country and not at all in the second one. While present for both countries, this phenomenon is more prevalent for Slovak artists, suggesting that Czech listeners are less willing to listen to Slovak music than Slovaks do to theirs.

\section{Used Technologies}
\begin{itemize}
We utilize a modern tech stack. The backbone of the project is Next.js and React, which were used to create the user interface. For the force graph, we used a premade component from D3. The final result is deployed on a private server.
\end{itemize}

\section{Lessons Learned}
Throughout this project, we learned a wide range of data visualization techniques and, more importantly, how crucial it is to view visualizations from the perspective of a new user. Simplicity and clarity are essential. The most important information should be easily visible, while secondary details should be accessible through interaction.

The data should be scaled properly. Sometimes it happened that the default graphs from the libraries we got didnt have proper minimums and maximums on the axes. This created discrepencies which our reviewers were suggesting we remove and so we did.

In many cases, smaller number of colors improves readability, with possibility to show details through interaction, labels, or tooltips.

\section{Contribution of Team Members}
Most of the things we did together, but the focus of the individual members was:

\begin{itemize}
\item Samuel: Implementation, data analysis
\item Matej: Research on visualization possibilities and design choices, report
\end{itemize}

\section{Screenshots}
\begin{figure}[H]
\centering
\includegraphics[width=0.4\textwidth]{all.png}
\caption{Artist collaboration network visualization}
\label{fig:network}
\end{figure}

\begin{figure}[H]
\centering
\includegraphics[width=0.4\textwidth]{top_ranks.png}
\caption{Detailed artist statistics and ranking trends}
\label{fig:detail}
\end{figure}

\section{Deployment}

\section{Deployment}

You can view the live application \href{http://i8c888wws0cw4wg04480wgkw.159.69.41.251.sslip.io/}{here}. Note: some images for tracks / artists might be missing due to restrictions of the spotify API.

\end{document}
